\documentclass{article}
\begin{document}
\title{Equation sets based on different parameters}
\maketitle
\section{Introduction}
This document contains sets of equations that are suitable to test the impact the length of the equation can have on the listener. We consider a few factors that may be important to assessing the listener capability to remember the equation. Equations in which each of these parameters vary are listed billow.  
\section{Equation variations based on number of variables}
$X=Y$\\
$X+Y=z$\\
$\frac{X+Y}{K}=\alpha$\\
$(X+Y)^K = 3*X^K+4*X^y-5Y^{K+X}$\\
$(X+Y)^{P+Q} = X^{P*Q}+Y^P*Q-P+\frac{Q}{Y}-\frac{P}{Q-X}$\\
$\frac{(P+X)*(Q-Y)}{(X+Y)^K} = \frac{P}{X+K}-Q*(\frac{K^x}{Y-P})$\\
\section{Equations with variations in number of terms}
$Y = X^2$\\
$X^Y+P^Q = \frac{X+Y}{P+Q}$\\
${P+Q+R}^K = P^K+Q-P*Q*R+P^{\frac{Q}{R}}$\\
$\frac{P}{K} + (P+K)*(P*K)*(P-K) = \frac{P*K}{K*P}+P^{K*P}$\\
$(A+B)^2 = \sqrt[4]{A+B}-\sqrt[2]{\frac{A}{B}}-\sqrt{(A+B)*(A-B)}+(A+B)*(\frac{A}{B}$\\
$P^Q+P^{R+Q}-P^R+Q+P^{R^{Q*P}} = P^{\frac{R+1}{Q}}-P^{R}+q^{\frac{1}{P}}$\\
\section{Equations with variation in number of repetitions of a variable}
$X+X^2 = x^3$\\
$x^2+X^3+x^9-X = X^{14}$\\
$X+\frac{X}{2}+\frac{X}{4}+\frac{X}{6+X} = \frac{16}{X}$\\
$K+K^X-K^{K+X}+K^{K^X}+K^{X*K} = X^{(K+X)*(X-K)}$\\
$\frac{X}{P}+\frac{X+1}{P+1}-\frac{X+2}{P+2}+\frac{3X}{3P} = \frac{(X+P)^3}{6^{P*X}}$\\
$\frac{X+Y+Z}{K} = X*\frac{Y}{K} + Y*\frac{Z}{K}+Z*\frac{X}{K}+\frac{Z+Y}{X*K}$\\
\end{document}