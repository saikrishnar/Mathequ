\documentclass{article}
\begin{document}
\title{Equations for testing the developed systems}
\author{Venkatesh Potluri}
\date{}
\maketitle
\section{Introduction}
We have hand-picked a few equations to test the system. We made sure that these equations are not standard equations. That is, a listener might not have come across these equations in any textbook. If the equations are arbitrary and random, The listener's previous knowledge will not have any effect on the results of the tests performed by him. How ever, the listener must be familiar with the mathematical concepts used in the equations. We aim to test the effectiveness of the outputs produced by our system, not the user's mathematical knowledge.
\section{Equations}
$1+2+3-5+4+2+3 = (3+2)*(1+1)$\\
\[ \lim_{x \to +\infty} \frac{3x^2 +7x^3}{x^2 +5x^4} = 3.\]\\
\[ \frac{\partial u}{\partial t}
   = h^2 \left( \frac{\partial^2 u}{\partial x^2}
      + \frac{\partial^2 u}{\partial y^2}
      + \frac{\partial^2 u}{\partial z^2} \right) \]\\
\[ \int_0^R \frac{2x\,dx}{1+x^2} = \log(1+R^2).\]\\
\[ \int_{x^2 + y^2 \leq R^2} f(x,y)\,dx\,dy
   = \int_{\theta=0}^{2\pi} \int_{r=0}^R
      f(r\cos\theta,r\sin\theta) r\,dr\,d\theta.\]\\
\[ \int_0^{+\infty} x^n e^{-x} \,dx = n!.\]\\
$(P+Q)^K+R = P^K*Q+Q^K*P+R^{P*Q}*K+\frac{P^Q*K+1}{R}$\\
$(P+Q)*(R+K) = (P+R)^Q-(K+R^Q)+\frac{R+Q^K}{(R+Q)^K+1}$\\
$\frac{X_1^K+X_2^K}{P^X_3*5^x_4}+E^X = e^{\frac{X_{K+1}+X_{K+2}}{(X+Y)}}$\\
\[ \int_{x^{2+y} + y^3 \geq R^2} Q^R+f(x,y)\,dx\,dy
   = 1+ \int_{\theta=0}^{2\pi} 2 * \int_{r=0}^R
      f(r\cos(\theta+2),r\sin\theta+2) r\,dr\,d\theta.\]\\
$\sqrt[P+Q]{A+K^P + A^{K+P}}=\frac{(K+P)(K-P)}{K*(P+K)}$\\
$\sum_{i=1}^{\infty} \frac{1}{i^2}+5i+\sqrt[3]{i+1} = \frac{\pi^2+4\pi^3+\sqrt[\pi+i]{9*\pi}}{6}$\\
$(\frac{X+Y}{K}+1)^3=\sqrt[3]{X}+\sqrt[3]{Y}+(X*Y)/3+\frac{X+Y}{3+K}+3$\\

\end{document}